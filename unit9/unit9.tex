\def\CTeXPreproc{Created by ctex v0.2.9, don't edit!}
%\documentclass{beamer}
\documentclass[%handout,
xcolor=pdftex]{beamer}
\mode<presentation> {
  \usetheme{Warsaw}
  \setbeamercovered{transparent}
}
\let\Tiny=\tiny
\usetheme{Singapore}
\usecolortheme{dolphin}
\usepackage{amsmath}
\usepackage{textcomp}
\usepackage{amssymb}
\usepackage{amsthm}
\usepackage{graphicx}
\usepackage{color}
\usepackage{lipsum}
\usepackage{hyperref}
\usepackage{multirow}
\usepackage{bm}
%\setbeamertemplate{headline}{}
\setbeamertemplate{footline}[page number]
\newcommand\Fontvi{\fontsize{9pt}{8}\selectfont}
\newcommand\Fontvii{\fontsize{7pt}{8}\selectfont}
\newcommand{\backupbegin}{
   \newcounter{finalframe}
   \setcounter{finalframe}{\value{framenumber}}
}
\newcommand{\backupend}{
   \setcounter{framenumber}{\value{finalframe}}
}\newtheorem{proposition}{Proposition}
\title{Unit 9: ARMA Models: MA(1)}
\author[STAT 5170: Applied Time Series, Unit 9]{Taylor R. Brown}
\institute{Department of Statistics, University of Virginia}
\date{Spring 2020}

\AtBeginSubsection[] {
  \begin{frame}<beamer>{Outline}
    \tableofcontents[currentsection,currentsubsection]
  \end{frame}
}

\begin{document}


\frame{\titlepage}


\begin{frame}
\frametitle{Readings for Unit 9}

Textbook chapter 3.1 (page 83 to 86).

\end{frame}


\begin{frame}
\frametitle{Last Unit}

\begin{enumerate}
\item AR(p) process
\item AR(1) process
\item AR(1) in terms of backshift operator
\item AR(1) and causality
\end{enumerate}

\end{frame}

\begin{frame}
\frametitle{This Unit}
\begin{enumerate}
\item Identifiability
\item MA(1) in terms of backshift operator
\item MA(1) and invertibility
\item ARMA model
\end{enumerate}
\end{frame}

\begin{frame}
\frametitle{Motivation}

In the previous unit, we noted a condition for an AR(1) model to be causal. In this unit, we will explore another issue, this time regarding MA models. This issue is called identifiability.

\end{frame}

\section{MA(1) and Invertibility}
\frame{\tableofcontents[currentsection]}

\begin{frame}
\frametitle{MA(q) Process}

The MA(q) model is defined as
$$
x_t=w_t + \theta_1 w_{t-1}+ \cdots + \theta_q w_{t-q}.
$$
We can rewrite this using backshift operator as
\begin{equation} \label{eq:MA}
x_t=\theta(B) w_t
\end{equation}

 where $\theta(B)=1+\theta_1 B+ \cdots +\theta_q B^q$ which is called the \textbf{moving average operator}. We already know that an MA(q) process is stationary.

\end{frame}

\begin{frame}
\frametitle{MA(1) Process}

Let's look at the $MA(1)$ model, i.e.
 $$
 x_t=w_t+ \theta w_{t-1}
 $$
 We can then calculate the autocovariance function.
 $$
 \gamma(0)=(1+\theta^2) \sigma_w^2
 $$
 and
 $$
 \gamma(1)=\theta \sigma_w^2
 $$
 and zero for larger lags.
 
 % Assume that we simulate from a model using $\theta=a$ and $\sigma_w^2=b$, then look at the autocovariance
 % $$ \gamma(0)=b+a^2 b, \quad  \gamma(1)=a b $$.

\end{frame}

\begin{frame}
\frametitle{Identifiability}

\textbf{Question}: Suppose we have another MA(1) model with $\tilde{\theta}=1/\theta$ and $\tilde{\sigma}_w^2=\theta^2 \sigma^2_w$, what is the autocovariance function for this model?

\vspace{40mm}
\end{frame}

\begin{frame}
\frametitle{Identifiability}

 Which one is ``correct''?  Both are correct: this is an \textbf{identifiability} problem as we have MA models that are not unique. We'll use the one that gives us an infinite order AR process.

\end{frame}


\begin{frame}
\frametitle{Identifiability}

 Rewrite the MA(1) model as
 $$
 w_t=-\theta w_{t-1}+x_t.
 $$
 This is just like the AR(1) that we saw last time but now the roles for $x_t, w_t$ are reversed.  So, we could write this as
\begin{eqnarray} \label{eq:invert}
 w_t &=& \sum_{j=0}^\infty (-\theta)^j x_{t-j} \nonumber\\
     &=& \sum_{j=0}^\infty (-\theta)^j B^j x_{t},
\end{eqnarray}
 provided that $|\theta|<1$.

\end{frame}

 \begin{frame}
\frametitle{MA(1) in Terms of Backshift Operator}
(\ref{eq:invert}) can be written as

\begin{equation}
 w_t=\sum_{j=0}^\infty \pi_j x_{t-j} = \pi(B) x_{t},
\end{equation}

where $\pi(B) = \sum_{j=0}^\infty \pi_j B^j$ and $\pi_j = (-\theta)^j$.

\end{frame}

\begin{frame}
\frametitle{Identifiability}

\textbf{Question}: What's an intuitive explanation as to why we want $|\theta|<1$ in (\ref{eq:invert})?

\vspace{60mm}

\end{frame}


\begin{frame}
\frametitle{Invertible MA(1)}

We could write an MA(1) as

\begin{eqnarray*}
 x_t &=& w_t + \theta w_{t-1} \\
     &=& w_t +\theta \sum_{j=0}^\infty (-\theta)^j x_{t-1-j}\\
     &=&  w_t - \sum_{j=1}^\infty (-\theta)^j x_{t-j}.
\end{eqnarray*}

 So, we have written the MA(1) as an infinite order AR process. We require $|\theta|<1$ for invertibility for an MA(1). An MA model which can be written as an infinite order AR process is called a \textbf{invertible} MA.  We will later extend and formalize a condition on the parameters $\theta_1, \theta_2, \cdots, \theta_q$ for an MA(q) process to be invertible.


\end{frame}



\section{ARMA Model}
\frame{\tableofcontents[currentsection]}

\begin{frame}
\frametitle{ARMA Model}

Recall that an AR(p) model is given by
 $$
 x_t=\phi_1 x_{t-1} +\phi_2 x_{t-2} + \cdots + \phi_p x_{t-p} + w_t
 $$
 or in terms of backshift operator
 $$
\phi(B) x_t =w_t, \quad \mbox{where} \quad \phi(B)=1-\phi_1 B-\phi_2 B^2-\cdots-\phi_p B^p.
$$

\end{frame}

\begin{frame}
\frametitle{ARMA Model}

An MA(q) model is given by
$$
x_t = w_t + \theta_1 w_{t-1} + \theta_2 w_{t-2} + \cdots + \theta_q w_{t-q}
$$
or in terms of backshift operator
$$
x_t =  \theta(B)w_t, \quad \mbox{where} \quad \theta(B)=1+\theta_1 B+\theta_2 B^2 +\cdots + \theta_q B^q.
$$

\end{frame}

\begin{frame}
\frametitle{ARMA Model}

$x_t$ is called an ARMA (p,q) model if $x_t$ is stationary and
can be written as
 $$
 \phi(B) x_t=\theta(B) w_t.
 $$

Thus far, we have seen a number of issues with the general definition of ARMA(p,q) models:

\begin{itemize}
\item Stationary AR models that depend on the \textbf{future}.
\item MA models that are not \textbf{unique} (identifiability issue).
\end{itemize}

We will next look at one more issue with ARMA(p,q) models before formalizing conditions for causality and invertibility.

\end{frame}



\end{document} 