\def\CTeXPreproc{Created by ctex v0.2.9, don't edit!}
%\documentclass{beamer}
\documentclass[%handout,
xcolor=pdftex]{beamer}
\mode<presentation> {
  \usetheme{Warsaw}
  \setbeamercovered{transparent}
}
\let\Tiny=\tiny
\usetheme{Singapore}
\usecolortheme{dolphin}
\usepackage{amsmath}
\usepackage{textcomp}
\usepackage{amssymb}
\usepackage{amsthm}
\usepackage{graphicx}
\usepackage{color}
\usepackage{lipsum}
\usepackage{hyperref}
\usepackage{multirow}
\usepackage{bm}
%\setbeamertemplate{headline}{}
\setbeamertemplate{footline}[page number]
\newcommand\Fontvi{\fontsize{9pt}{8}\selectfont}
\newcommand\Fontvii{\fontsize{7pt}{8}\selectfont}
\newcommand{\backupbegin}{
   \newcounter{finalframe}
   \setcounter{finalframe}{\value{framenumber}}
}
\newcommand{\backupend}{
   \setcounter{framenumber}{\value{finalframe}}
}\newtheorem{proposition}{Proposition}
\title{Unit 11: ARMA Autocorrelation and Partial Autocorrelation Functions}
\author[STAT 5170: Applied Time Series, Unit 11]{Jeffrey Woo}
\institute{Department of Statistics, University of Virginia}
\date{Spring 2020}

\AtBeginSubsection[] {
  \begin{frame}<beamer>{Outline}
    \tableofcontents[currentsection,currentsubsection]
  \end{frame}
}

\begin{document}


\frame{\titlepage}


\begin{frame}
\frametitle{Readings for Unit 11}

Textbook chapter 3.3.

\end{frame}


\begin{frame}
\frametitle{Last Unit}
\begin{enumerate}
\item ARMA(p,q)
\item Condition for causality
\item Condition for invertibility
\end{enumerate}
\end{frame}


\begin{frame}
\frametitle{This Unit}
\begin{enumerate}
\item ACF for MA(q)
\item ACF for Causal ARMA(p,q)
\item Partial Autocorrelation Function (PACF)
\end{enumerate}
\end{frame}


\begin{frame}
\frametitle{Motivation}

In this unit we will study the autocorrelation and partial autocorrelation
 functions for ARMA processes.

\end{frame}

\section{ACF for MA(q) Processes}
\frame{\tableofcontents[currentsection]}

\begin{frame}
\frametitle{MA(q) Process}

 Let's start with an MA($q$) process
 $$
 x_t = w_t + \theta_1 w_{t-1} + \theta_2 w_{t-2} + \cdots + \theta_q w_{t-q} = \sum^q_{j=0} \theta_j w_{t-j},
 $$
where we have written $\theta_0=1$. Then
$$
E(x_t) =\sum^q_{j=0} \theta_j E(w_{t-j})=0.
$$

\end{frame}

\begin{frame}
\frametitle{Autocovariance for MA(q)}

The autocovariance function is
\begin{eqnarray*}
\gamma(h) = cov(x_t,x_{t+h}) &=& E \Big [ \sum^q_{j=0} \theta_j w_{t-j} \sum^q_{j'=0} \theta_{j'} w_{t+h-j'} \Big ] \cr
&=& \sum^q_{j=0} \sum^q_{j'=0} \theta_j \theta_{j'} E(w_{t-j} w_{t+h-j'}).
\end{eqnarray*}

\end{frame}

\begin{frame}
\frametitle{Autocovariance for MA(q)}

Recall that $E(w_s w_t) = \sigma^2_w$ if $s=t$ and $E(w_s
w_t)=0$ otherwise. So we have
\begin{eqnarray}\label{eq:1}
\gamma(h) = \left \{ \begin{array}{ll}
\sigma^2_w \sum^{q-h}_{j=0} \theta_j \theta_{j+h}, & 0 \le h\le q, \\
0, & h \ge q+1.
\end{array} \right.
\end{eqnarray}


\end{frame}

\begin{frame}
\frametitle{ACF for MA(q)}

Recall that $\gamma(h)=\gamma(-h)$, so we will only need the
values for $h\ge 0$. Dividing $\gamma(h)$ by $\gamma(0)$ in
(\ref{eq:1}), we obtain the autocorrelation function (ACF) of
an MA($q$) model
\begin{eqnarray}\label{eq:2}
\rho(h) = \left \{ \begin{array}{ll}
\frac{\sum^{q-h}_{j=0} \theta_j \theta_{j+h}}{1+\theta^2_1+\cdots+\theta^2_q}, & 0 \le h\le q, \\
0, & h \ge q+1.
\end{array} \right.
\end{eqnarray}


\end{frame}

\section{ACF for Causal ARMA(p,q) Processes}
\frame{\tableofcontents[currentsection]}

\begin{frame}
\frametitle{ACF for Causal ARMA(p,q)}

We have seen in (\ref{eq:2}), for MA($q$) models, the ACF will
be zero for lags greater than $q$. Moreover, because
$\theta_q\ne 0$,
$\rho(q)=\theta_0\theta_q/(1+\theta_1^2+\cdots+\theta^2_q)\ne
0$. Thus, the ACF provides information
about the order of the dependence for a MA model. How about
ARMA or AR models?

\end{frame}

\begin{frame}
\frametitle{Causal ARMA(p,q)}

Now we discuss causal ARMA($p,q$) model
$$
\phi(B) x_t = \theta(B) w_t,
$$
where the roots of $\phi(z)$ are outside the unit circle. Since
the model is causal, we have the MA($\infty$) representation
\begin{eqnarray}\label{eq:3}
x_t = \sum^\infty_{j=0} \psi_j w_{t-j},  \quad \mbox{where} \quad
\psi(z) = \frac{\theta(z)}{\phi(z)} = \sum^\infty_{j=0} \psi_j z^j.
\end{eqnarray}

\end{frame}

\begin{frame}
\frametitle{Autocovariance for Causal ARMA(p,q)}

It follows  that $E(x_t)=0$ and by (\ref{eq:1}), the
autocovariance function of $x_t$ is given by
$$
\gamma(h) =cov(x_t,x_{t+h})= \sigma^2_w \sum^\infty_{j=0} \psi_j \psi_{j+h},
\quad h\ge 0.
$$

\end{frame}

%\begin{frame}
%\frametitle{$\psi$-weights for Causal ARMA(p,q)}

%\textbf{Question}: Given functions $\theta(z)$ and $\phi(z)$, how to
%obtain the coefficients $\psi_j$ in (\ref{eq:3})?

%\vspace{40mm}

%\end{frame}



\begin{frame}
\frametitle{Autocovariance for Causal AR(1)}

To motivate the idea, consider a causal AR(1) model $x_t=\phi_1
x_{t-1}+w_t$.  Then

\vspace{40mm}

\end{frame}

\begin{frame}
\frametitle{Autocovariance for Causal AR(1)}

Here, we have used causality: $x_{t-2}$ only depends on
$w_{t-2},w_{t-3},\ldots$ and hence is independent of $w_{t-1}$
and $w_t$. Consequently, $cov(w_{t-1},x_{t-2})=0$ and
$cov(w_t,x_{t-2})=0$. The ACF for an AR(1) model is very
different from an MA(1) model whose ACF is zero at lag 2.

\end{frame}

\begin{frame}
\frametitle{Autocovariance for Causal AR(1)}

\textbf{Question}: Why is $\gamma(2)\ne 0$ for an AR(1) process?

\vspace{50mm}

\end{frame}

\begin{frame}
\frametitle{Removing Connection}

Consider the following,

\vspace{40mm}

because of causality. Thus, after removing the
connection $x_{t-1}$, the covariance between $x_t$ and
$x_{t-2}$ is zero.

\end{frame}

\begin{frame}
\frametitle{Notation}

One way to remove linear connections is through \textbf{linear
regression}. Let $\hat{x}_{t+h}$ denote the regression of $x_{t+h}$ on
$\{x_{t+h-1},x_{t+h-2},\ldots,x_{t+1}\}$, which we write as
\begin{equation}
\hat{x}_{t+h} = \beta_1 x_{t+h-1} + \beta_2 x_{t+h-2} + \cdots + \beta_{h-1} x_{t+1}.
\end{equation}
Here we do not include the intercept assuming the mean of $x_t$
is zero. Otherwise, replace $x_t$ with $x_t - \mu_x$.

\end{frame}

\begin{frame}
\frametitle{Notation}

In addition, let $\hat{x}_t$ denote the regression of
$x_t$ on $\{x_{t+1},x_{t+2},\ldots,x_{t+h-1}\}$, then
\begin{equation}
\hat{x}_t = \beta_1 x_{t+1} + \beta_2 x_{t+2} + \cdots + \beta_{h-1} x_{t+h-1}.
\end{equation}

\end{frame}

\section{Partial Autocorrelation Function}
\frame{\tableofcontents[currentsection]}

\begin{frame}
\frametitle{Partial Autocorrelation Function}

The {\bf partial autocorrelation
function (PACF)} of a stationary process $x_t$, denoted by
$\phi_{hh}$, for $h=1,2,\ldots$, is

\begin{equation}
\phi_{11} = corr(x_{t+1},x_t) = \rho(1)
\end{equation}
and
\begin{equation}
\phi_{hh}=corr(x_{t+h}-\hat{x}_{t+h}, x_t-\hat{x}_t), \quad h\ge 2.
\end{equation}

Note that, the PACF, $\phi_{hh}$ is the correlation between
$x_{t+h}$ and $x_{t}$ with the linear dependence of
$\{x_{t+1},\cdots,x_{t+h-1}\}$, on each, removed.

\end{frame}

\begin{frame}
\frametitle{Partial Autocorrelation Function}

Now we consider PACF of a causal AR(1) model: $x_t=\phi
x_{t-1}+w_t$, with $|\phi|<1$. By definition,
$\phi_{11}=corr(x_1,x_0)=\rho(1)=\phi$. To calculate
$\phi_{22}$, consider the regression of $x_{t+2}$ on $x_{t+1}$, say
$\hat{x}_{t+2}=\beta x_{t+1}$.

\end{frame}

\begin{frame}
\frametitle{Partial Autocorrelation Function}

Recall in linear regression we seek to minimize the error sum of squares (SSE). In this setting, we seek to minimize

\vspace{50mm}

\end{frame}

\begin{frame}
\frametitle{Partial Autocorrelation Function}

Next, consider the regression of $x_t$ on $x_{t+1}$, say
$\hat{x}_t=\beta x_{t+1}$. We choose $\beta$ to minimize

\vspace{50mm}

\end{frame}

\begin{frame}
\frametitle{Partial Autocorrelation Function}

In general, for a causal AR($p$) model $x_h=\sum^p_{j=1}\phi_j
x_{h-j}+w_h$. When $h>p$, the regression of $x_h$ on
$x_{h-1},\ldots,x_1$ is
$$
\hat{x}_{h} = \sum^p_{j=1} \phi_j x_{h-j}.
$$

\end{frame}

\begin{frame}
\frametitle{Partial Autocorrelation Function}

Thus, when $h>p$, by causality,
$$
\phi_{hh} = corr(x_h-\hat{x}_{h},x_0-\hat{x}_{0}) = corr(w_h,x_0-\hat{x}_{0}) = 0.
$$

\end{frame}

\begin{frame}
\frametitle{Summary}

\begin{itemize}
\item The ACF of MA($q$) model cuts at lag $q$. The PACF of an AR($p$) model cuts at lag $p$.
\item Identification of an MA(q) model is best done with ACF; identification of an AR(p) model is best done with PACF.
\item The PACF between $x_t$ and $x_{t-h}$ is the conditional correlation between $x_t$ and $x_{t-h}$. It is conditional on $x_{t-h+1}, \cdots, x_{t-1}$, the set of observations that come \textbf{between time $t$ and $t-h$}.
\end{itemize}

\end{frame}

\section{Worked Examples}
\frame{\tableofcontents[currentsection]}

\begin{frame}
\frametitle{ACF and PACF of Causal AR(2)}

\includegraphics[width=100mm, height=80mm]{ar.pdf}

\end{frame}

\begin{frame}
\frametitle{ACF and PACF of Invertible MA(2)}

\includegraphics[width=100mm, height=80mm]{ma.pdf}

\end{frame}

\begin{frame}
\frametitle{ACF and PACF of Causal and Invertible ARMA(2,2)}

\includegraphics[width=100mm, height=80mm]{arma.pdf}

\end{frame}

\begin{frame}
\frametitle{ACF and PACF of Causal AR and Invertible MA}

(From page 99, Table 3.1 of text)

\begin{center}
\begin{tabular}{cccc}
\hline \\
 & \textbf{AR(p)} & \textbf{MA(q)} & \textbf{ARMA(p,q)} \\
\hline \\
ACF & Decay & 0 after lag $q$ & Decay\\
PACF & 0 after lag $p$ & Decay & Decay\\
\hline
\end{tabular}
\end{center}

\end{frame}

\begin{frame}
\frametitle{Fish Population Example}

This time series from ``recruit.dat" contains data on fish population in the central Pacific Ocean. The numbers represent the number of new fish in the years 1950-1987. \textbf{Question}: Based on the ACF and PACF plots, what process do you think is most likely to describe this time series?

\end{frame}

\begin{frame}
\frametitle{Fish Population Example}

\includegraphics[width=100mm, height=80mm]{recruit.pdf}

\end{frame}

\end{document} 