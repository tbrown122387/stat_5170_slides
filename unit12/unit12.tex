\def\CTeXPreproc{Created by ctex v0.2.9, don't edit!}
%\documentclass{beamer}
\documentclass[%handout,
xcolor=pdftex]{beamer}
\mode<presentation> {
  \usetheme{Warsaw}
  \setbeamercovered{transparent}
}
\let\Tiny=\tiny
\usetheme{Singapore}
\usecolortheme{dolphin}
\usepackage{amsmath}
\usepackage{textcomp}
\usepackage{amssymb}
\usepackage{amsthm}
\usepackage{graphicx}
\usepackage{color}
\usepackage{lipsum}
\usepackage{hyperref}
\usepackage{multirow}
\usepackage{mathrsfs}
\usepackage{bm}
%\setbeamertemplate{headline}{}
\setbeamertemplate{footline}[page number]
\newcommand\Fontvi{\fontsize{9pt}{8}\selectfont}
\newcommand\Fontvii{\fontsize{7pt}{8}\selectfont}
\newcommand{\backupbegin}{
   \newcounter{finalframe}
   \setcounter{finalframe}{\value{framenumber}}
}
\newcommand{\backupend}{
   \setcounter{framenumber}{\value{finalframe}}
}\newtheorem{proposition}{Proposition}
\title{Unit 12: Forecasting}
\author[STAT 5170: Applied Time Series, Unit 12]{Jeffrey Woo}
\institute{Department of Statistics, University of Virginia}
\date{Spring 2020} %ADD BLP AND PACF?

\AtBeginSubsection[] {
  \begin{frame}<beamer>{Outline}
    \tableofcontents[currentsection,currentsubsection]
  \end{frame}
}

\begin{document}


\frame{\titlepage}


\begin{frame}
\frametitle{Readings for Unit 12}


Textbook chapter 3.4 (skip pages 103 to 106).

\end{frame}



\begin{frame}
\frametitle{Last Unit}
\begin{enumerate}
\item ACF for MA(q)
\item ACF for Causal ARMA(p,q)
\item Partial Autocorrelation Function
\end{enumerate}
\end{frame}

\begin{frame}
\frametitle{This Unit}
\begin{enumerate}
\item Best linear predictor
\item ARMA forecasting
\end{enumerate}
\end{frame}


\begin{frame}
\frametitle{Motivation}

In this unit, we explore forecasting: predicting future values of a time series based on observed data.

\end{frame}

\section{Forecasting for Stationary Processes}
\frame{\tableofcontents[currentsection]}

\begin{frame}
\frametitle{Forecasting}

In forecasting, the goal is to predict future values of a time series, $x_{n+m}$, based on the observed data $\mathbf{x} = \{x_n, x_{n-1}, \cdots, x_1 \}$. In this unit, we assume $\{x_t\}$ is stationary.

\end{frame}

\begin{frame}
\frametitle{Forecasting}

The minimum mean square error predictor of $x_{n+m}$ is

\begin{equation}
x_{n+m}^n = \mbox{E}(x_{n+m} | \mathbf{x})
\end{equation}

as the conditional expectation minimizes the mean square error $\mbox{E}\left[x_{n+m} - g(\mathbf{x}) \right]^2$, where $g(\mathbf{x})$ is a function of the observations.

\end{frame}

\begin{frame}
\frametitle{Forecasting}

First, we restrict our attention to predictors that are linear functions of the observations, i.e.

\begin{equation} \label{eq:BLP}
x_{n+m}^n = \alpha_0 + \sum_{j=1}^n \alpha_j x_j
\end{equation}

where $\alpha_0, \alpha_1, \cdots, \alpha_n \in \mathbb{R}$. Linear predictors of the form (\ref{eq:BLP}) that minimize the mean square prediction error are called \textbf{best linear predictors (BLP)}.\\

\vspace{5mm}
Linear prediction depends on on the second-order moments of the process, which can be estimated from the data.

\end{frame}

\begin{frame}
\frametitle{Projection Theorem}

\begin{theorem}
\label{projection}
Let $\mathcal{M}$ be a closed subspace of the Hilbert space $\mathcal{H}$ and let $y$ be an element in $\mathcal{H}$. Then, $y$ can be uniquely represented as $y = \hat{y} + z$ where $\hat{y} \in \mathcal{M}$ and $z$ is orthogonal to $\mathcal{M}$. Therefore, for any $w \in \mathcal{M}$,

\begin{itemize}
\item $\lVert y-w \rVert \geq \lVert y-\hat{y} \rVert$ and
\item $<z,w> = 0$.
\end{itemize}
\end{theorem}

%$\hat{y}$ is the projection of $y$ on $\mathcal{M}$.

\end{frame}

\begin{frame}
\frametitle{Projection Theorem}


\end{frame}

\begin{frame}
\frametitle{Projection Theorem: Linear Prediction}

Given $1, x_1, x_2, \cdots, x_n \in \{X: \mbox{E}(X^2) < \infty \}$, choose $\alpha_0, \alpha_1, \cdots, \alpha_n \in \mathbb{R}$ so that $U = \alpha_0 + \sum_{j=1}^n \alpha_j x_j$ minimizes $\mbox{E}(x_{n+m} - U)^2$.\\
\vspace{5mm}

Note that: $\mathcal{M} = \{U = \alpha_0 + \sum_{j=1}^n \alpha_j x_j: \alpha_j \in \mathbb{R} \} = \bar{sp}\{1,x_1, \cdots, x_n \}$ and $y=x_{n+m}$.
\end{frame}

\begin{frame}
\frametitle{Projection Theorem: Linear Prediction}

Let $x_{n+m}^n$ denote the best linear predictor, i.e.

$$
\lVert x_{n+m}^n - x_{n+m} \rVert^2 \leq \lVert U - x_{n+m} \rVert^2
$$

for all $U \in \mathcal{M}$. The projection theorem implies\\

\vspace{40mm}

\end{frame}

\begin{frame}
\frametitle{Projection Theorem: Linear Prediction}

\begin{itemize}
\item The prediction errors $x_{n+m}^n - x_{n+m}$ are orthogonal to the prediction variables $(1,x_1, \cdots, x_n)$.
\item Orthogonality of prediction error and 1 implies we can \textbf{subtract the mean} from all variables $x_{n+m}$ and $x_i$.
\item Therefore, we typically assume $\mu=0$ for forecasting.
\end{itemize}


\end{frame}

\begin{frame}
\frametitle{BLP for Stationary Process}

Given $x_1, \cdots, x_n$, the best linear predictor for stationary processes, $x_{n+m}^n = \alpha_0 + \sum_{j=1}^n \alpha_j x_j$, of $x_{n+m}$, for $m \geq 1$, is found by solving

\begin{equation} \label{eq:predict_eqns}
\mbox{E}\left[ (x_{n+m} - x_{n+m}^n)x_k \right] = 0 \text{ for } k=0, 1,\cdots,n,
\end{equation}

where $x_0 =1$, for $\alpha_0,\alpha_1,\cdots,\alpha_n$. The equations (\ref{eq:predict_eqns}) are called the \textbf{prediction equations}.

\end{frame}

\begin{frame}
\frametitle{One-Step-Ahead Linear Prediction}

Consider one-step-ahead prediction. Given $x_1, \cdots, x_n$, we want to forecast $x_{n+1}$. The BLP takes the form

\begin{equation}
x_{n+1}^n = \phi_{n1}x_n + \phi_{n2} x_{n-1} + \cdots + \phi_{nn}x_1.
\end{equation}

Therefore, the prediction equations (\ref{eq:predict_eqns}) become\\

\vspace{50mm}

\end{frame}

\begin{frame}
\frametitle{One-Step-Ahead Linear Prediction}

In matrix form:

\vspace{70mm}

\end{frame}


\begin{frame}
\frametitle{One-Step-Ahead Linear Prediction}

The mean square one-step-ahead prediction error is

\begin{eqnarray}
P_{n+1}^n &=& \mbox{E}(x_{n+1} - x_{n+1}^n)^2 \nonumber \\
          &=& \nonumber \\
          &=& \nonumber \\
          &=& \nonumber \\
          &=& \nonumber \\
          &=&
\end{eqnarray}
\end{frame}

\begin{frame}
\frametitle{Prediction Intervals}

Construct prediction interval:

$$
x_{n+1}^n \pm 1.96 \sqrt{P_{n+1}^n}.
$$


for Gaussian processes. The prediction error has distribution $N(0,P_{n+1}^n)$.

\end{frame}


\section{ARMA Forecasting}
\frame{\tableofcontents[currentsection]}

\begin{frame}
\frametitle{ARMA Forecasting}

Let's consider an ARMA model that is causal and invertible
$$
\phi(B)  X_t = \theta(B) w_t,
$$
where
$$
\phi(B) = 1-\phi_1 B - \cdots - \phi_p B^p \quad
\mbox{and} \quad
\theta(B) = 1+\theta_1 B + \cdots+\theta_q B^q.
$$
\end{frame}

\begin{frame}
\frametitle{ARMA Forecasting}

By causality and invertibility, we have
\begin{eqnarray}\label{eq:1}
 x_{n+m}=\sum_{j=0}^\infty \psi_j w_{n+m-j},\quad \ \psi_0=1,
\end{eqnarray}

where $\psi(z) = \frac{\theta(z)}{\phi(z)} = \sum^\infty_{j=0} \psi_j z^j$,  and
 \begin{eqnarray}\label{eq:2}
 w_{n+m}=\sum_{j=0}^\infty \pi_j x_{n+m-j}, \quad \ \pi_0=1,
\end{eqnarray}

where $\pi(z) = \frac{\phi(z)}{\theta(z)} = \sum^\infty_{j=0} \pi_j z^j$.
\end{frame}


\begin{frame}
\frametitle{ARMA Forecasting}

Given the past information $x_{n},x_{n-1},\ldots$, we are
interested in predicting $x_{n+m}$. We use $\widetilde{x}_{n+m}=E(x_{n+m}|x_n,x_{n-1},...) $,
the conditional expectation of $x_{n+m}$ given all the past
$x_n,x_{n-1},\ldots,$ to forecast $x_{n+m}$.

\end{frame}

\begin{frame}
\frametitle{ARMA Forecasting}

Note also that
$$
E(w_t|x_n,x_{n-1},...)=0
$$
for $t>n$  because of causality. For $t \leq n$, $w_t$ is
determined by $x_t,x_{t-1},\ldots,$ which are included in
$x_n,x_{n-1},\ldots$. Thus,
$$
E(w_t|x_n,x_{n-1},...)=w_t.
$$

\end{frame}

\begin{frame}
\frametitle{ARMA Forecasting}

So

\begin{eqnarray}\label{eq:causal_white}
E(w_t|x_n,x_{n-1},...) = \left \{ \begin{array}{ll}
0, & t > n, \\
w_t, & t \leq n.
\end{array} \right.
\end{eqnarray}

\end{frame}


\begin{frame}
\frametitle{ARMA Forecasting}

Now, we take the infinite AR representation (\ref{eq:2}) and
take the conditional expectation (conditioning on $x_n,
x_{n-1},...$) on both sides of (\ref{eq:2}) to get

\vspace{40mm}

\end{frame}

\begin{frame}
\frametitle{ARMA Forecasting}

This leads to
\begin{eqnarray}\label{eq:3}
\widetilde{x}_{n+m} =- \sum_{j=1}^{m-1} \pi_j \widetilde{x}_{n+m-j}-\sum_{j=m}^\infty \pi_j x_{n+m-j}.
\end{eqnarray}
Letting $m=1$ in (\ref{eq:3}), we have
\begin{eqnarray*}
\widetilde{x}_{n+1} = -\sum^\infty_{j=1} \pi_j x_{n+1-j} = - \sum^\infty_{j'=0} \pi_{j'+1} x_{n-j'}.
\end{eqnarray*}
So, start by finding $\widetilde{x}_{n+1}$ and then recursively use
(\ref{eq:3}) to find the later $\widetilde{x}_{n+m}$. This is called the \textbf{one-step ahead predictor}.

\end{frame}

\begin{frame}
\frametitle{Worked Example}

We have an AR(2) model $x_t = \phi_1 x_{t-1} + \phi_2 x_{t-2} + w_t$. Suppose we have observations up to the $n$th term, i.e. $x_n, x_{n-1}, \cdots$. So we wish to use the observed data and estimated AR(2) model to forecast the value of $x_{n+1}$ and $x_{n+2}$. Writing out these two values, we have

\begin{eqnarray*}
x_{n+1} &=& \phi_1 x_{n} + \phi_2 x_{n-1} + w_{n+1}, \\
x_{n+2} &=& \phi_1 x_{n+1} + \phi_2 x_{n} + w_{n+2}.
\end{eqnarray*}

\end{frame}

\begin{frame}
\frametitle{Worked Example}

To forecast $x_{n+1}$, we use the observed values of $x_{n}$ and $x_{n-1}$ and replace $w_{n+1}$ by its expected value of 0.\\
\vspace{5mm}
Forecasting $x_{n+2}$ poses a challenge, since it requires the unobserved value of $x_{n+1}$. We use the forecasted value of $x_{n+1}$.

\end{frame}

\begin{frame}
\frametitle{Framework in Forecasting}

In general, the forecasting procedure for an ARMA(p,q) model is as follows:

\begin{itemize}
\item For any $w_j$ with $1 \leq j \leq n$, use the sample residual at time $j$.
\item For any $w_j$ with $j>n$, use the expected value of $w_j$, which is 0.
\item For any $x_j$ with $1 \leq j \leq n$, use the observed value of $x_j$.
\item For any $x_j$ with $j>n$, use the forecasted value of $x_j$.
\end{itemize}

\end{frame}

\begin{frame}
\frametitle{Prediction Error}

We use the infinite MA representation (\ref{eq:1}) and write

\vspace{50mm}
\end{frame}

\begin{frame}
\frametitle{Prediction Error}

Therefore, the mean-squared prediction error, or variance of the difference between the forecasted value and the true value at time $n+m$ is
\begin{equation} \label{eq:pred_error}
P_{n+m}^n=E(x_{n+m} -\widetilde{x}_{n+m})^2=\sigma_w^2 \sum_{j=0}^{m-1} \psi^2_j,
\end{equation}

and the standard error of the forecast error at time $n+m$ is

\begin{equation} \label{eq:se_error}
\sqrt{\hat{\sigma_w^2} \sum_{j=0}^{m-1} \psi^2_j}.
\end{equation}

\end{frame}

\begin{frame}
\frametitle{Prediction Error}

\textbf{Question}: Write out the standard error of the forecast error for $m=1$ and $m=2$.

\vspace{50mm}

\end{frame}



\begin{frame}
\frametitle{Prediction Error}

Notice that as $m$ gets larger--i.e. as we predict further into
the future, this is \textbf{increasing} but essentially asymptotes. This
means that you are getting essentially a constant prediction
interval after a certain distance into the future, as if we do not know what was going on previously.

\end{frame}

\begin{frame}
\frametitle{Prediction Error}

Also, for fixed sample size $n$, the prediction errors are correlated. For $h \geq 1$,

$$
E\left\{(x_{n+m} -\widetilde{x}_{n+m})(x_{n+m+h} -\widetilde{x}_{n+m+h})\right\}=\sigma^2 \sum_{j=0}^{m-1} \psi_j \psi_{j+k}.
$$

\end{frame}

\begin{frame}
\frametitle{Prediction Interval}

For Gaussian processes, the $95\%$ prediction interval for $x_{n+m}$, the future value of the series at time $n+m$ is

\begin{equation}
x_{n+m}^{n} \pm 1.96 \sqrt{\hat{\sigma_w^2} \sum_{j=0}^{m-1} \psi_j^2}.
\end{equation}

\end{frame}

\begin{frame}
\frametitle{Worked Example}

\textbf{Question}: We have an AR(1) model $x_t = 40 + 0.6 x_{t-1} + w_t$. Suppose we have $n=100$ observations, $\hat{\sigma_w^2} = 1$ and $x_{100} = 80$. We wish to forecast the values at times 101 and 102.

\vspace{40mm}



\end{frame}

\begin{frame}
\frametitle{Worked Example}


\end{frame}

\end{document} 